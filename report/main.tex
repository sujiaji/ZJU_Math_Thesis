\documentclass{beamer}  
\usepackage[UTF8]{ctex}
\usepackage{amsmath}  
\usepackage{mathdots}  
\usepackage{graphicx}  
\usepackage{float}  
\usepackage{multirow}  
\usetheme{Berkeley}  
%可选主题 除 Warsaw 外有:
% Antibes Bergen Berkeley Berlin
% Boadilla Copenhagen Darmstadt Dresden
% Frankfurt Goettingen Hannover Ilmenau
% Juanlespins Madrid Malmoe Marburg
% Montpellier Paloalto Pittsburgh Rochester
% Singapore
    \begin{document}   
    %%%%%%%%%%%%%%%%%   开始页   %%%%%%%%%%%%%%%%%%%%%%%
        \title{beamer 实例}  
        \subtitle{这有一个小标题}
        \date{\today} 
        \author{苏佳骥}
        \institute{浙江大学数学科学学院}
        \frame{\titlepage}

    %%%%%%%%%%%%%%%%%  目录页    %%%%%%%%%%%%%%%%%%%%%%%%%%
        \frame{\frametitle{目录}\tableofcontents} 
        % 下面为定义章节的代码,在需要使用的地方添加即可改变目录
        %\section{1}  
        %\subsection{subsection name}
        %\subsubsection{sub-subsection name}
        %\section{2}  
        %\section{3}   
        

    %%%%%%%%%%%%%%%%%  正文页实例   %%%%%%%%%%%%%%%%%%%%%%%%%%
    %有标题
        \begin{frame}
            \frametitle{帧标题}
        内容
        \end{frame}
    %无标题的无序列表
    \begin{frame}[plain]
        空白帧没有标题
        \begin{itemize}
            \item The first item
            \item The second item
            \item The third item
            \item The fourth item
            \end{itemize}
    \end{frame}
    %有序列表
    \begin{frame}[plain]
        \begin{enumerate}
            \item The first item
            \item The second item
            \item The third item
            \item The fourth item
            \end{enumerate}
    \end{frame}
    %描述列表
    \begin{frame}[plain]
        \begin{description}
            \item[First Item] Description of first item
            \item[Second Item] Description of second item
            \item[Third Item] Description of third item
            \item[描述] Description of forth item
            \end{description}
    \end{frame}
    %区块
     \begin{frame}[plain]
        \begin{block}{勾股定理}  
        直角三角形的斜边的平方等于两直角边的平方和。  
        可以用符号语言表述为:设直角三角形ABC,其中$\angle C=90^\circ$则有  
        \begin{equation}  
            AB^2=BC^2+AC^2  
        \end{equation}  
        \end{block}  
    \end{frame}
    
    \section{第一章}
    \begin{frame}
        \frametitle{第一章}
        \subsection {第一节}
        第一节
    \end{frame}
    \end{document}
