\addcontentsline{toc}{part}{指导教师对文献综述和开题报告具体要求}
\cfoot{~}
\begin{spacing}{2}
    \noindent \hwfs \zihao{4} \textbf{一、题目:脑电波与神经回路计算模型}\\
    \noindent \hwfs \zihao{4} \textbf{二、指导教师对文献综述和开题报告的具体要求:}\\
    \textbf{文献综述要求:}\par

    {\color{red}{教务网对应文件下载}}

    要求查阅与毕业论文(设计)相关的文献8—10篇以上。(其中外文文献不少于3—5篇),译文
    (译文可作为文献的一部分)和文献综述(包括国内外现状、研究方向、进展情况、存在问题、参考依据)要求字数各3000字以上,
    文献综述内容要切题。在整个研究和撰写过程中,应注意认真严谨的学风,严守学术道德准则,杜绝学术不端。\par
    根据阅读的国内外文献撰写文献综述报告,要求根据主题展开,文献综述内容要切题,包括\\
    \setlength{\parindent}{1em}
    (1)简述与脑电波与神经回路计算模型相关的近几年的论文的研究内容和相关进展,掌握当前该领域的研究前沿,分析你毕业论文研究的内容与这些论文的差异和相关。\\
    (2) 分析与具体要研究的脑电波与神经回路计算模型相关的论文所采用的研究方法,包括理论方法和模拟方法等,简述这些方法的优劣和你的思考。\\
    (3)简述各参考文献的创新性、存在的问题或未能解决的问题。\\
    (4) 要求翻译其中的一篇外文文献,结构完整,语句通顺。\\
    \par \noindent \textbf{开题报告要求:}\\
    \setlength{\parindent}{1em}
    (1) 分析具体要研究的脑电波与神经回路计算模型的意义(包括理论意义和实际意义,并分析课题与本专业的关系)\\
    (2)根据文献综述分析课题的研究背景(即要解决什么问题?这些问题在其他文献中有没有讨论过?本文所讨论问题的角度与已有参考文献中所涉及的问题的差异,课题的主要创新点是什么?\\
    (3) 选题的可行性分析(一般从数据的可获得性、研究技术的可行性等方面去说明)。\\
    (4) 主要研究内容(这部分要展开写,主要包括理论研究内容、实证分析内容和调研内容等)。\\
    (5)根据研究的内容写出具体的实施计划。\\
    (6)明确论文最后预期结果。\\
    
\end{spacing}

\vfill
\fangsong \zihao{-4} \raggedleft \textbf
{
    指导教师(签名)\underline{\rule{3.5cm}{0pt}} \\ 
    ~\\    
    \raggedleft 年 \qquad 月 \qquad 日
}

\raggedright 
\newpage