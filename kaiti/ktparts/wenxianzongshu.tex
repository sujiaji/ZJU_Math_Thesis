\section{一、文献综述}
\begin{spacing}{1.5}
        \zihao{-4} \fangsong \raggedright 
        \setlength{\parindent}{2em}
\subsection{1~背景介绍}         
小四号1.5倍行距,小四号1.5倍行距,小四号1.5倍行距,小四号1.5倍行距,小四号1.5倍行距,小四号1.5倍行距,小四号1.5倍行距,小四号1.5倍行距,
小四号1.5倍行距,小四号1.5倍行距,小四号1.5倍行距,小四号1.5倍行距,小四号1.5倍行距,小四号1.5倍行距,小四号1.5倍行距,小四号1.5倍行距, 
\subsection{2~国内外研究现状}
        \subsubsection{2.1~研究方向及进展}
        小四号1.5倍行距,小四号1.5倍行距,小四号1.5倍行距,小四号1.5倍行距,小四号1.5倍行距,小四号1.5倍行距,小四号1.5倍行距,小四号1.5倍行距,
    小四号1.5倍行距,小四号1.5倍行距,小四号1.5倍行距,小四号1.5倍行距,小四号1.5倍行距,小四号1.5倍行距,小四号1.5倍行距,小四号1.5倍行距, 
\subsection{3~研究展望}

小四号1.5倍行距,小四号1.5倍行距,小四号1.5倍行距,小四号1.5倍行距,小四号1.5倍行距,小四号1.5倍行距,小四号1.5倍行距,小四号1.5倍行距,
小四号1.5倍行距,小四号1.5倍行距,小四号1.5倍行距,小四号1.5倍行距,小四号1.5倍行距,小四号1.5倍行距,小四号1.5倍行距,小四号1.5倍行距, 
\end{spacing}

\subsection{4~参考文献}
\begin{spacing}{1.1}
        \noindent \hangafter=1 \setlength{\hangindent}{2.5em} [1]	唐章宏, 袁建生. 用有限元法计算媒质各向异性真实头模型脑电正问题[J]. 中国生物医学工程学报, 2003, 22(3):208-214.
        
        \noindent \hangafter=1 \setlength{\hangindent}{2.5em} [2]	尧德中, 饶妮妮, 傅世敏,等. 脑电逆问题的延时相关阵子空间分解算法[J]. 电子学报, 2000, 28(4):135-138.

        \noindent \hangafter=1 \setlength{\hangindent}{2.5em}  [3]	姚远. 脑电计算中有限元真实头模型的构造研究[D]. 浙江大学, 2006.

        \noindent \hangafter=1 \setlength{\hangindent}{2.5em} [4]	刘君. 脑电计算中基于医学图像的真实头有限元模型构造研究[D]. 浙江大学电气工程学院 浙江大学, 2007.

        \noindent \hangafter=1 \setlength{\hangindent}{2.5em} [5]	李璟, 王琨, 刘君,等. 利用有限差分法计算真实头模型脑电正问题[J]. 传感技术学报, 2007, 20(8):1736-1741.

        \noindent \hangafter=1 \setlength{\hangindent}{2.5em}  [6]	何娟. MEG、EEG正问题的数值模拟及其反问题研究[D]. 上海师范大学, 2013.
\end{spacing}

\newpage