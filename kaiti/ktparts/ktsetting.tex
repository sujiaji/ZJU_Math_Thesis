\usepackage{amsmath}
\usepackage{amssymb}
\usepackage{extarrows}
\usepackage{indentfirst}
\usepackage{geometry}
\usepackage{setspace}
\usepackage{fancyhdr} 
\usepackage{fontspec}
\usepackage{titlesec} 
\usepackage{titletoc}
\usepackage{multirow}
\usepackage{booktabs}
\usepackage{pdfpages}
\usepackage{natbib}
\pagestyle{fancy}
%%%%%%% 字体相关 %%%%%
\newcommand{\tnewroman}{\fontspec{Times New Roman}} %Times New Roman  tnewroman
\setCJKfamilyfont{hwfs}{STFangsong}   %华文仿宋 hwfs 
\newcommand{\hwfs}{\CJKfamily{hwfs}} 

%%%% 页边距  %%%%%%%%%
\geometry{top=72pt,right=90pt,bottom=72pt,left=90pt} 

%%%%%%% 标题格式 %%%%%%%
    %%%%%   目录部分   %%%%%%
    
\renewcommand{\contentsname}{\centerline{ \hwfs \zihao{3} 目\quad 录}}
        % \part 部分相关
\titlecontents{part}[0pt]
    {\addvspace{4pt}\filright  \hwfs \zihao{-4}}
    {\contentspush{\thecontentslabel  }}{}{}
        % \section 相关
\titlecontents{section}[0pt]{\addvspace{4pt} \zihao{-4}  \bf \hwfs }
    {}{}
    {\titlerule*[6pt]{.}\contentspage}
        %\subsection 相关
\titlecontents{subsection}[10pt]{\addvspace{4pt} \zihao{-4}  \bf \hwfs }
    {}{}
    {\titlerule*[6pt]{.}\contentspage}
        %\subsubsection 相关
\titlecontents{subsubsection}[10pt]{\addvspace{4pt} \zihao{-4} \hwfs }
    {}{}
    {\titlerule*[6pt]{.}\contentspage}

    %%%%  文章内部  %%%%%
\titleformat{\section}%设置section的样式
    {\centering \zihao{3} \bf \hwfs}%居中,3号字,加粗
    {}%手动标号
    {0pt}%sep label和title之间的水平距离
    {}%标题前没有内容

\titleformat{\subsection}%设置subsection的样式
    {\raggedright \zihao{-3} \bf \hwfs }%右对齐,小3号字,加粗
    {}%手动标号
    {0pt}%sep label和title之间的水平距离
    {}%标题前没有内容

\titleformat{\subsubsection}%设置subsubsection的样式
    {\raggedright \zihao{4} \bf \hwfs }%右对齐,4号字,加粗
    {}%手动标号
    {0pt}%sep label和title之间的水平距离
    {}%标题前没有内容

%%%%% 将当前页页眉重置为右侧页眉
\newcommand{\resetoddpageheadings}{  
    \ifodd\value{page}
        {
        \fancyhead{}            % 清空fancy设定
        \fancyhead[RO]{\zihao{-5} \biaoti}     %奇数页 右侧 页眉信息
        \fancyhead[LE]{\zihao{-5} 浙江大学本科生毕业论文(设计)}    %偶数页 左侧 页眉信息
        }
        \else
        {
        \fancyhead{}            % 清空fancy设定
        \fancyhead[RE]{\zihao{-5} \biaoti}     %奇数页 右侧 页眉信息
        \fancyhead[LO]{\zihao{-5} 浙江大学本科生毕业论文(设计)}    %偶数页 左侧 页眉信息
        }
    \fi
    }