\section{二、开题报告}
\begin{spacing}{1.5}
    \zihao{-4} \fangsong \raggedright 
    \setlength{\parindent}{2em}
\subsection{1~问题提出的背景}
    \subsubsection{1.1~背景介绍}         
    小四号1.5倍行距,小四号1.5倍行距,小四号1.5倍行距,小四号1.5倍行距,小四号1.5倍行距,小四号1.5倍行距,小四号1.5倍行距,小四号1.5倍行距,
    
    \subsubsection{1.2~本研究的意义和目的}         
    小四号1.5倍行距,小四号1.5倍行距,小四号1.5倍行距,小四号1.5倍行距,小四号1.5倍行距,小四号1.5倍行距,小四号1.5倍行距,小四号1.5倍行距,


\subsection{2~论文的主要内容和技术路线}
        \subsubsection{2.1~主要研究内容}
        小四号1.5倍行距,小四号1.5倍行距,小四号1.5倍行距,小四号1.5倍行距,小四号1.5倍行距,小四号1.5倍行距,小四号1.5倍行距,小四号1.5倍行距,
        小四号1.5倍行距,小四号1.5倍行距,小四号1.5倍行距,小四号1.5倍行距,小四号1.5倍行距,小四号1.5倍行距,小四号1.5倍行距,小四号1.5倍行距,

        \subsubsection{2.2~技术路线}
        小四号1.5倍行距,小四号1.5倍行距,小四号1.5倍行距,小四号1.5倍行距,小四号1.5倍行距,小四号1.5倍行距,小四号1.5倍行距,小四号1.5倍行距,


        \subsubsection{2.3~可行性分析}
        小四号1.5倍行距,小四号1.5倍行距,小四号1.5倍行距,小四号1.5倍行距,小四号1.5倍行距,小四号1.5倍行距,小四号1.5倍行距,小四号1.5倍行距,

\subsection{3~研究计划进度安排及预期目标}
        \subsubsection{3.1~进度安排}
        (1)1月-1月14日:导师下达任务书,对进度、文献和开题提出要求;\\
        (2)1月15日-1月23日:学生确认任务书,对确定的课题搜集相关文献资料,了解问题的背景、应用、研究历史与现状。从中确定论文最终题目。 \\
        (3)1月24日-3月2日:对确定的题目进一步展开学习,包括所必需的基础知识及近几年涉及此问题的文章。初步撰写并完成开题报告、文献综述,并提交导师审核。\\
        (4)3月3日-3月6日:组织开题,每位学生准备10分钟左右的答辩;\\
        (5)3月7日-4月11日:将定稿的开题报告、文献综述、外文翻译稿上传至教务系统。做中期检查报告。\\
        (6)4月12日-5月12日:完成论文初稿,进行论文稿的修改并最终完成,向导师提交论文终稿。\\
        (7)5月13日-5月15日:导师评阅,学生提交导师填写评语和签字的“毕业论文考核表”及符合规范格式要求的送审论文。\\
        (8)5月16日-5月21日:毕业论文专家评阅.\\
        (9)5月22日-5月24日:评阅结果有修改意见的,根据评阅意见对论文进行修改。\\
        (10)5月24日-5月30日:组织毕业论文答辩。提交最终版毕业论文,并将论文上传至教务系统。

        \subsubsection{3.2~预期目标}

        小四号1.5倍行距,小四号1.5倍行距,小四号1.5倍行距,小四号1.5倍行距,小四号1.5倍行距,小四号1.5倍行距,小四号1.5倍行距,小四号1.5倍行距,


\subsection{4~参考文献}
\end{spacing}

\begin{spacing}{1.1}
        \noindent \hangafter=1 \setlength{\hangindent}{2.5em} [1]	唐章宏, 袁建生. 用有限元法计算媒质各向异性真实头模型脑电正问题[J]. 中国生物医学工程学报, 2003, 22(3):208-214.
        
        \noindent \hangafter=1 \setlength{\hangindent}{2.5em} [2]	尧德中, 饶妮妮, 傅世敏,等. 脑电逆问题的延时相关阵子空间分解算法[J]. 电子学报, 2000, 28(4):135-138.

        \noindent \hangafter=1 \setlength{\hangindent}{2.5em}  [3]	姚远. 脑电计算中有限元真实头模型的构造研究[D]. 浙江大学, 2006.

        \noindent \hangafter=1 \setlength{\hangindent}{2.5em} [4]	刘君. 脑电计算中基于医学图像的真实头有限元模型构造研究[D]. 浙江大学电气工程学院 浙江大学, 2007.

        \noindent \hangafter=1 \setlength{\hangindent}{2.5em} [5]	李璟, 王琨, 刘君,等. 利用有限差分法计算真实头模型脑电正问题[J]. 传感技术学报, 2007, 20(8):1736-1741.

        \noindent \hangafter=1 \setlength{\hangindent}{2.5em}  [6]	何娟. MEG、EEG正问题的数值模拟及其反问题研究[D]. 上海师范大学, 2013.

        \noindent \hangafter=1 \setlength{\hangindent}{2.5em}  [7]	Kim D, Seo H, Kim H I, et al. Computational study on subdural cortical stimulation - the influence of the head geometry, anisotropic conductivity, and electrode configuration[J]. Plos One, 2014, 9(9):e108028.

        \noindent \hangafter=1 \setlength{\hangindent}{2.5em}  [8]	Park J S, Chung M S, Hwang S B, et al. Visible Korean Human: its techniques and applications.[J]. Clinical Anatomy, 2006, 19(3):216-24.

        \noindent \hangafter=1 \setlength{\hangindent}{2.5em}  [9]	Windhoff M, Opitz A, Thielscher A. Electric field calculations in brain stimulation based on finite elements: an optimized processing pipeline for the generation and usage of accurate individual head models.[J]. Human Brain Mapping, 2013, 34(4):923–935.
\end{spacing}
\newpage