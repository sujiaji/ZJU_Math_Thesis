%//                            _ooOoo_
%//                           o8888888o
%//                           88" . "88
%//                           (| -_- |)
%//                            O\ = /O
%//                        ____/`---'\____
%//                      .   ' \\| |// `.
%//                       / \\||| : |||// \
%//                     / _||||| -:- |||||- \
%//                       | | \\\ - /// | |
%//                     | \_| ''\---/'' | |
%//                      \ .-\__ `-` ___/-. /
%//                   ___`. .' /--.--\ `. . __
%//                ."" '< `.___\_<|>_/___.' >'"".
%//               | | : `- \`.;`\ _ /`;.`/ - ` : | |
%//                 \ \ `-. \_ __\ /__ _/ .-` / /
%//         ======`-.____`-.___\_____/___.-`____.-'======
%//                            `=---='

\documentclass[a4paper, 12pt,twoside]{ctexart}
% 苏佳骥 2018年5月18日
% 使用XeLaTeX → bibtex → XeLaTeX → XeLaTeX 过程编译,若提示交叉引用问题,则再重复一次编译
%%%%%  tex中全篇设置双面打印,页眉信息可自行调整,实际打印时自行判断单双面打印的范围
% 使用字体 华文仿宋, Times New Roman, 仿宋
\usepackage{amsmath}
\usepackage{amssymb}
\usepackage{amsthm}
\usepackage{cite}
\usepackage{extarrows}
\usepackage{indentfirst}
\usepackage{geometry}
\usepackage{multirow}
\usepackage{setspace}
\usepackage{fancyhdr} 
\usepackage{fontspec}
\usepackage{titlesec} 
\usepackage{titletoc}
\usepackage{pdfpages}
\usepackage{bm}
\usepackage{listings} 
\renewcommand{\thepart}{第\chinese{part}部分}
\renewcommand{\contentsname}{\centerline{ \hwfs \zihao{3} 目\quad 录}}
\pagestyle{fancy}
%%%%%%% 字体相关 %%%%%
\newcommand{\tnewroman}{\fontspec{Times New Roman}} %Times New Roman  tnewroman
\setCJKfamilyfont{hwxk}{STXingkai}   %华文行楷  hwxk  
\newcommand{\hwxk}{\CJKfamily{hwxk}} 
\setCJKfamilyfont{hwfs}{STFangsong}   %华文仿宋 hwfs 
\newcommand{\hwfs}{\CJKfamily{hwfs}} 
\newfontfamily\fangsong{FangSong}       %仿宋
\newcommand{\erhao}{\fontsize{21pt}{\baselineskip}\selectfont}

%%%% 页边距  %%%%%%%%%

\geometry{top=72pt,right=90pt,bottom=72pt,left=90pt} 

%%%%%%% 标题格式 %%%%%%%
    %%%%%   目录部分   %%%%%%
        % \part 部分相关
\titlecontents{part}[0pt]
    {\addvspace{1pt}\filright \bf \hwfs \zihao{4}}
    {\contentspush{\thecontentslabel  }}{}{}
        % \section 相关
\titlecontents{section}[0pt]{\zihao{-4} \hwfs }
    {\contentspush{\thecontentslabel ~}}{}
    {\titlerule*[6pt]{.}\contentspage}
        %\subsection 相关
\titlecontents{subsection}[0pt]{\zihao{-4} \hwfs }
    {\contentspush{\thecontentslabel ~}}{}
    {\titlerule*[6pt]{.}\contentspage}
        %\subsubsection 相关
\titlecontents{subsubsection}[0pt]{\zihao{-4} \hwfs }
    {\contentspush{\thecontentslabel ~}}{}
    {\titlerule*[6pt]{.}\contentspage}

    %%%%  文章内部  %%%%%
\titleformat{\part}%设置part的样式
    {\vspace{9cm}\centering \fontsize{48}{2} \heiti} %标签 48 磅 居中
    {\thepart }
    {0pt}%sep label和title之间的水平距离
    { \\ \centering \zihao{-0} \bf  \hwfs } % 标题仿宋GB, 加粗,小初居中

\titleformat{\section}%设置section的样式
    {\raggedright \zihao{3} \bfseries \hwfs}%右对齐,3号字,加粗
    {\thesection~}%标号后面有空格
    {0pt}%sep label和title之间的水平距离
    {}%标题前没有内容

\titleformat{\subsection}%设置subsection的样式
    {\raggedright \zihao{-3} \bfseries \hwfs }%右对齐,小3号字,加粗
    {\thesubsection~}%标号后面有个空格
    {0pt}%sep label和title之间的水平距离
    {}%标题前没有内容

\titleformat{\subsubsection}%设置subsubsection的样式
    {\raggedright \zihao{4} \bf \hwfs }%右对齐,4号字,加粗
    {\thesubsubsection~}%标号后面有个空格
    {0pt}%sep label和title之间的水平距离
    {}%标题前没有内容

%%%% 图、表、公式编号
\renewcommand{\thefigure}{\thesection.\arabic{figure}}
\renewcommand{\thetable}{\thesection.\arabic{table}}
\renewcommand\theequation{~ \arabic{section} - \arabic{equation} ~}
\renewcommand\arraystretch{1.5}
\renewcommand{\refname}{\vspace{-1cm}}

%%%%% 将当前页页眉重置为右侧页眉
\newcommand{\resetoddpageheadings}{  
    \ifodd\value{page}
        {
        \fancyhead{}            % 清空fancy设定
        \fancyhead[RO]{\zihao{-5} \biaoti}     %奇数页 右侧 页眉信息
        \fancyhead[LE]{\zihao{-5} 浙江大学本科生毕业论文(设计)}    %偶数页 左侧 页眉信息
        }
        \else
        {
        \fancyhead{}            % 清空fancy设定
        \fancyhead[RE]{\zihao{-5} \biaoti}     %奇数页 右侧 页眉信息
        \fancyhead[LO]{\zihao{-5} 浙江大学本科生毕业论文(设计)}    %偶数页 左侧 页眉信息
        }
    \fi
    }
%%%%%%%%%%%%  定义论文封面相关信息  %%%%%%%%%%%%%%%%
\newcommand\leixing{本~科~生~毕~业~论~文(设~计)}
\newcommand\biaoti{论文模板}
\newcommand\xuehao{314010xxxx ~ 苏佳骥}
\newcommand\jiaoshi{x ~ x}  %指导教师
\newcommand\banji{2014 ~ 统计学}
\newcommand\xueyuan{数学科学学院}
\newcommand\tijiaoriqi{2018年5月15日} %提交日期
\fancyhead{}
\fancyhead[RO]{\zihao{-5} \biaoti}      %奇数页 页眉信息
\fancyhead[RE]{\zihao{-5} 浙江大学本科生毕业论文(设计)}      %偶数页 页眉信息
\begin{document}   
    %%%%%%%%%%  创建封面   %%%%%%%%%%%%%%%%
    \begin{titlepage}
    \begin{flushright}
        \label{OLE_LINK1}                         
        \textbf{涉密论文} $\square{}$  ~\textbf{公开论文} $\square{}$
    \end{flushright}    
    \begin{figure}[h]
        \begin{center}
        \includegraphics[width=10.8cm,height=2.7cm]{pics/c1.jpg} 
        \end{center}
    \end{figure}
    \vspace{0.5cm}
    \centering
    \zihao{1} \heiti \textbf{\leixing}\\   
    \vspace{0.5cm}
    \begin{figure}[h]
        \begin{center}
        \includegraphics[width=3.49cm,height=3.41cm]{pics/c2.jpg}
        \end{center}
    \end{figure}
        \vspace{0.5cm}
    \begin{table}[thp]
        \zihao{3} \hwfs \bfseries
        \centering        
        \renewcommand\arraystretch{1}
        \begin{tabular}{cc}
            \makebox[5em][s]{题\hspace{\fill}目} & 
                \textbf{\biaoti} \\ \cline{2-2}\\
                \\
                \\
            \makebox[5em][s]{姓\hspace{\fill}名\hspace{\fill}与\hspace{\fill}学\hspace{\fill}号} & 
                \xuehao  \\\cline{2-2} \\
            \makebox[5em][s]{指\hspace{\fill}导\hspace{\fill}教\hspace{\fill}师} & 
                \jiaoshi  \\ \cline{2-2} \\
            \makebox[5em][s]{年\hspace{\fill}级\hspace{\fill}与\hspace{\fill}专\hspace{\fill}业} & 
                \banji \\ \cline{2-2} \\
            \makebox[5em][s]{所\hspace{\fill}在\hspace{\fill}学\hspace{\fill}院} & 
                \xueyuan \\ \cline{2-2}\\ 
                \\
            \makebox[5em][s]{提\hspace{\fill}交\hspace{\fill}日\hspace{\fill}期} & 
                \tijiaoriqi  \\\cline{2-2}
        \end{tabular}
    \end{table}
\end{titlepage}     
    %%%%%%%%%%% 承诺书 %%%%%%%%%
    \pagenumbering{Roman}
\par
\begin{center}
    \textbf{{\zihao{3} \hwfs 浙江大学本科生毕业论文(设计)承诺书}}
\end{center}
\vspace{0.5cm}
\begin{spacing}{1.8}
    {\hwfs \zihao{4}
            1.本人郑重地承诺所呈交的毕业论文(设计),是在指导教师的指导下严格按照学校和学院有关规定完成的。

            2.本人在毕业论文(设计)中除了文中特别加以标注和致谢的地方外,论文中不包含其他人已经发表或撰写过的研究成果,也不包含为获得 \underline{\bf ~浙江大学~ } 或其他教育机构的学位或证书而使用过的材料。

            3.与我一同工作的同志对本研究所做的任何贡献均已在论文中作了明确的说明并表示谢意。

            4.本人承诺在毕业论文(设计)工作过程中没有伪造数据等行为。

            5.若在本毕业论文(设计)中有侵犯任何方面知识产权的行为,由本人承担相应的法律责任。

            6.本人完全了解 \underline{\bf ~浙江大学~ } 有权保留并向有关部门或机构送交本论文(设计)的复印件和磁盘,允许本论文(设计)被查阅和借阅。本人授权 \underline{\bf ~浙江大学~ } 可以将本论文(设计)的全部或部分内容编入有关数据库进行检索和传播,可以采用影印、缩印或扫描等复制手段保存、汇编本论文(设计)。
            \\
            \begin{table}[thp]    
                \hwfs   
                \renewcommand\arraystretch{1.5}
                \begin{tabular}{lcl}
                   作者签名: & & 导师签名: \\ 
                   \\
                   签字日期:\qquad 年 \quad 月 \quad 日 & \hspace{2.75cm} & 签字日期:\qquad 年 \quad 月 \quad 日
                \end{tabular}
            \end{table}
            
    }
\end{spacing}
\newpage    
    %%%%%%% 致谢 %%%%%%%
    \resetoddpageheadings  %% 将页眉重新设置为奇数页情形以便单面打印
    \begin{center}
    {\zihao{3} \hwfs \textbf{致 ~ 谢} }
\end{center}

\begin{spacing}{1.5}
    { \zihao{-4} \fangsong
    仿宋 小四号1.5倍行距,

   仿宋 小四号1.5倍行距,
    }
\end{spacing}
\newpage
    %%%%%%%%%%   中文摘要&关键词      %%%%%%%%%%%%%%%
    \resetoddpageheadings
    \input{parts/zhaiyao.tex}
    %%%%%%%%%%   英文摘要&关键词      %%%%%%%%%%%%%%%
    \resetoddpageheadings
    \begin{center}
    {\zihao{3}  \textbf{\tnewroman Abstract (\hwfs 英文)} }
\end{center}

\begin{spacing}{1.5}
    { \zihao{-4} \tnewroman
   timesnewRoman 小四号1.5倍行距,timesnewRoman 小四号1.5倍行距,timesnewRoman 小四号1.5倍行距,timesnewRoman 小四号1.5倍行距,timesnewRoman 小四号1.5倍行距,

   timesnewRoman 小四号1.5倍行距,
    
    \noindent \textbf{Key words: } xxx; xxx; xxx
    }
\end{spacing}
\newpage   
    %%%%%%% 目录 %%%%%
    \resetoddpageheadings
    \thispagestyle{fancy}
    \tableofcontents
    %%%% 代码不会写,手动填写页码
    {\zihao{-4} \hwfs \noindent 文献综述和开题报告封面}\\ 
    {\zihao{-4} \hwfs \noindent 指导教师对文献综述和开题报告具体内容要求}\\    
    {\zihao{-4} \hwfs \noindent 目录\titlerule*[6pt]{.} \quad I}\\    
    {\zihao{-4} \hwfs \noindent 一、文献综述\titlerule*[6pt]{.} \quad 1}\\    
    {\zihao{-4} \hwfs \noindent 二、开题报告\titlerule*[6pt]{.} \quad 5}\\    
    {\zihao{-4} \hwfs \noindent 三、外文翻译\titlerule*[6pt]{.} \quad 9}\\    
    {\zihao{-4} \hwfs \noindent 四、外文原文\titlerule*[6pt]{.} \quad 16}\\      
    {\zihao{-4} \hwfs \noindent 《浙江大学本科生文献综述和开题报告考核表》}
    \newpage
    %%%%% 毕业论文 %%%%%%
    \thispagestyle{empty} 
    \part{ 毕业论文(设计)}
    \pagenumbering{arabic} 
    \newpage
    \setcounter{page}{1}
    %%% 导入parts/zhengwen.tex 中的论文正文
    \begin{spacing}{1.5}
{ \zihao{-4} \hwfs
%%%%%%%%%%%%%%%
    \section{章标题}
    \subsection{节标题}
    华文仿宋 小四号1.5倍行距,
    华文仿宋 小四号1.5倍行距,

    华文仿宋 小四号1.5倍行距,
    \subsection{节标题}
    
    华文仿宋 小四号1.5倍行距,
    \subsection{节标题}

    \begin{figure}[ht]
        \centering
        \includegraphics[scale=0.3]{pics/c1.jpg}
        \caption{ \zihao{5} \bf 图片示例}   
        \label{123} %%交叉索引 用 \ref{123} 引用
        \vspace{0.3cm}
    \end{figure}


    

 
    插入图片
    \begin{figure}[ht]
        \centering
        \includegraphics[scale=0.6]{pics/c1.jpg}
        \caption{ \zihao{5} \bf 图片及标题示例}    
        \vspace{0.4cm}
    \end{figure}
        
    插入表格
    \begin{table}[thp]
        \caption{ \zihao{5} \bf 表格示例}
        \zihao{5} \hwfs
        \vspace{0.2cm}
        \centering        
        \begin{tabular}{c|c|c}% 通过添加 | 来表示是否需要绘制竖线
            \hline  % 在表格最上方绘制横线
            xxx & XXX & xxx\\
            \hline  %在第一行和第二行之间绘制横线
            xxx & XXX & xxx\\
            \hline % 在表格最下方绘制横线        
        \end{tabular}
        \vspace{0.4cm}
   \end{table}

    公式示例
    \begin{equation}
        \nabla\cdot(\vec\sigma \nabla V)=0  in \Omega
   \end{equation}

\clearpage
}
\end{spacing}
\newpage
    %%%%% 参考文献 %%%%
    \resetoddpageheadings
    \addcontentsline{toc}{section}{参考文献}
\begin{center}
    {\zihao{3} \hwfs \textbf{参考文献} }
    \nocite{*}
    \bibliography{ref}
    \bibliographystyle{unsrt}
\end{center}

\newpage
    %%%%%%%%%%   附录      %%%%%%%%%%%%%%%
    \resetoddpageheadings
    \addcontentsline{toc}{section}{附录}
\begin{center}
    {\zihao{3} \hwfs\textbf{附~录} }
\end{center}

\begin{spacing}{1.5}
    { \zihao{-4} \fangsong
    MATLAB求解使用代码
    }
    \begin{lstlisting}[language=MATLAB] 
 % writen by SU Jiaji @ April 2018, mainly for my thesis 
 about solving eeg inverse problem


 end
\end{lstlisting} 
\end{spacing}
\newpage 
    %%%%%%%%%%   作者简历      %%%%%%%%%%%%%%%
    \resetoddpageheadings
    \addcontentsline{toc}{section}{作者简历}
\begin{center}
    {\zihao{3} \hwfs\textbf{作者简历} }
\end{center}

\begin{spacing}{1.5}
    { \zihao{-4} \fangsong
    姓名:苏佳骥 \quad 
    性别:男\quad   
    民族:满 \quad  
    出生年月:19xx-xx-xx 
    
    籍贯:xx省xx市

    2011.09-2014.07   xxxx

    2014.09-2018.07   xxxx

    获奖情况:xxxx

    参加项目:xx

    发表的学术论文:xx
    
    }
\end{spacing}
\newpage 
    %%%%%%%%%%   任务书      %%%%%%%%%%%%%%%
    \resetoddpageheadings
    \addcontentsline{toc}{section}{《浙江大学本科生毕业论文(设计)任务书》}

\begin{center}
    {\zihao{2} \hwxk\textbf{本科生毕业论文(设计)任务书} }
\end{center}

 \begin{spacing}{1.5}
        \noindent \hwfs \zihao{4} \textbf{一、题目:{\color{red}{填写}}}\\
        \noindent \hwfs \zihao{4} \textbf{二、指导教师对毕业论文(设计)的进度安排及任务要求:}
        \par
        { \setlength{\parindent}{1em} \zihao{-4} \fangsong 

        {\color{red}{教务网对应内容}}
        (1)1月——1月14日:导师下达任务书,对进度、文献和开题提出要求;

        (2)1月15日——1月23日:学生确认任务书,对确定的课题搜集相关文献资料,了解问题的背景、应用、研究历史与现状。从中确定论文最终题目。
        
        (3)1月24日——3月2日:对确定的题目进一步展开学习,包括所必需的基础知识及近几年涉及此问题的文章。初步撰写并完成开题报告、文献综述,并提交导师审核。
        
        (4)3月3日——3月6日:组织开题,每位学生准备10分钟左右的答辩;
        
        (5)3月7日——4月11日:将定稿的开题报告、文献综述、外文翻译稿上传至教务系统。做中期检查报告。
        
        (6)4月12日——5月12日:完成论文初稿,进行论文稿的修改并最终完成,向导师提交论文终稿。
        
        (7)5月13日——5月15日:导师评阅,学生提交导师填写评语和签字的“毕业论文考核表”及符合规范格式要求的送审论文。
        
        (8)5月16日——5月21日:毕业论文专家评阅。
        
        (9)5月22日——5月24日:评阅结果有修改意见的,根据评阅意见对论文进行修改。
        
        (10)5月24日——5月30日:组织毕业论文答辩。提交最终版毕业论文,并将论文上传至教务系统。
        
        \vspace{2cm}
    
    \noindent 起讫日期 2018  年 1  月  14 日 至 2018  年  5 月  30 日\\

    \fangsong \zihao{-4} \raggedleft \textbf
        {
            指导教师(签名)\underline{\rule{3.5cm}{0pt}}职称\underline{\rule{2.5cm}{0pt}} \\
            \vspace{0.5cm}    
            \raggedleft 年 \qquad 月 \qquad 日
        }

        \raggedright 

        \noindent \hwfs \zihao{4} \textbf{三、系或研究所审核意见:}\\
            \qquad 同意该计划!
        \vspace{2cm}
        
        \fangsong \zihao{-4} \raggedleft \textbf
        {
            负责人(签名)\underline{\rule{3.5cm}{0pt}} \\ 
            \vspace{0.5cm}      
            \raggedleft 年 \qquad 月 \qquad 日
        }

        \raggedright
        \newpage
        }
\end{spacing} 
    %%%%%%%%%%  考核表      %%%%%%%%%%%%%%%
    \resetoddpageheadings
    \addcontentsline{toc}{section}{《浙江大学本科生毕业论文(设计)考核表》}
\cfoot{~}
\centering \hwfs \zihao{3} \textbf{毕~业~论~文(设计)\quad  考~核}\\ 
\vspace{0.5cm}
\raggedright \zihao{4} \hwfs \textbf{一、指导教师对毕业论文(设计)的评语:}
\vspace{0.1cm}

{ \setlength{\parindent}{2em} \zihao{-4} \hwfs

{\color{red}{教务网对应内容}}

}

\vspace{2cm}
\fangsong \zihao{-4} \raggedleft \textbf
        {
            指导教师(签名)\underline{\rule{3.5cm}{0pt}}\\
            \vspace{0.3cm}    
            \raggedleft 年 \qquad 月 \qquad 日
        }

\vspace{1cm }
\raggedright \zihao{4} \hwfs \textbf{ 二、 答辩小组对毕业论文(设计)的答辩评语及总评成绩:}
\vspace{0.1cm}

{ \setlength{\parindent}{2em} \zihao{-4} \hwfs
{\color{red}{教务网对应内容}}}

\vfill
\renewcommand\arraystretch{1.2}
  \begin{table}[htbp]
    \hwfs

    \resizebox{\textwidth}{!}{
    \begin{tabular}{|c|c|c|c|c|c|}
    \hline
      \multirow{2}[2]{*}{\textbf{成绩比例}} & \multicolumn{1}{p{5.5em}|}{\textbf{文献综述}} & \multicolumn{1}{p{5.285em}|}{\textbf{开题报告}} & \multicolumn{1}{p{4.855em}|}{\textbf{外文翻译}} & \multicolumn{1}{p{13.43em}|}{\textbf{毕业论文(设计)质量及答辩}} & \multicolumn{1}{c|}{\multirow{2}[2]{*}{\textbf{总评成绩}}} \\
      \multicolumn{1}{|c|}{} & \multicolumn{1}{p{5.5em}|}{\textbf{占(10\%)}} & \multicolumn{1}{p{5.285em}|}{\textbf{占(15\%)}} & \multicolumn{1}{p{4.855em}|}{\textbf{占(5\%)}} & \multicolumn{1}{p{13.43em}|}{\textbf{占(70\%)}} &  \\
      \hline
      \textbf{分} &       &       &       &       &  \\      
      \textbf{值} &       &       &       &       &  \\
      \hline
      \end{tabular}
    }

  \end{table}%

  \par 
  \vspace{0.3cm}
\fangsong \zihao{-4} \raggedleft \bfseries
{
    答辩小组负责人(签名)\underline{\rule{3.5cm}{0pt}} \\ 
    ~\\
    年 \qquad 月 \qquad 日
}

\newpage 
    \thispagestyle{empty} 
    \part{文献综述和开题报告}
    \newpage     
\end{document}