\addcontentsline{toc}{section}{《浙江大学本科生毕业论文(设计)任务书》}

\begin{center}
    {\zihao{2} \hwxk\textbf{本科生毕业论文(设计)任务书} }
\end{center}

 \begin{spacing}{1.5}
        \noindent \hwfs \zihao{4} \textbf{一、题目:{\color{red}{填写}}}\\
        \noindent \hwfs \zihao{4} \textbf{二、指导教师对毕业论文(设计)的进度安排及任务要求:}
        \par
        { \setlength{\parindent}{1em} \zihao{-4} \fangsong 

        {\color{red}{教务网对应内容}}
        (1)1月——1月14日:导师下达任务书,对进度、文献和开题提出要求;

        (2)1月15日——1月23日:学生确认任务书,对确定的课题搜集相关文献资料,了解问题的背景、应用、研究历史与现状。从中确定论文最终题目。
        
        (3)1月24日——3月2日:对确定的题目进一步展开学习,包括所必需的基础知识及近几年涉及此问题的文章。初步撰写并完成开题报告、文献综述,并提交导师审核。
        
        (4)3月3日——3月6日:组织开题,每位学生准备10分钟左右的答辩;
        
        (5)3月7日——4月11日:将定稿的开题报告、文献综述、外文翻译稿上传至教务系统。做中期检查报告。
        
        (6)4月12日——5月12日:完成论文初稿,进行论文稿的修改并最终完成,向导师提交论文终稿。
        
        (7)5月13日——5月15日:导师评阅,学生提交导师填写评语和签字的“毕业论文考核表”及符合规范格式要求的送审论文。
        
        (8)5月16日——5月21日:毕业论文专家评阅。
        
        (9)5月22日——5月24日:评阅结果有修改意见的,根据评阅意见对论文进行修改。
        
        (10)5月24日——5月30日:组织毕业论文答辩。提交最终版毕业论文,并将论文上传至教务系统。
        
        \vspace{2cm}
    
    \noindent 起讫日期 2018  年 1  月  14 日 至 2018  年  5 月  30 日\\

    \fangsong \zihao{-4} \raggedleft \textbf
        {
            指导教师(签名)\underline{\rule{3.5cm}{0pt}}职称\underline{\rule{2.5cm}{0pt}} \\
            \vspace{0.5cm}    
            \raggedleft 年 \qquad 月 \qquad 日
        }

        \raggedright 

        \noindent \hwfs \zihao{4} \textbf{三、系或研究所审核意见:}\\
            \qquad 同意该计划!
        \vspace{2cm}
        
        \fangsong \zihao{-4} \raggedleft \textbf
        {
            负责人(签名)\underline{\rule{3.5cm}{0pt}} \\ 
            \vspace{0.5cm}      
            \raggedleft 年 \qquad 月 \qquad 日
        }

        \raggedright
        \newpage
        }
\end{spacing}