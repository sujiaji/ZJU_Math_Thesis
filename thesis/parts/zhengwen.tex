\begin{spacing}{1.5}
{ \zihao{-4} \hwfs
%%%%%%%%%%%%%%%
    \section{章标题}
    \subsection{节标题}
    华文仿宋 小四号1.5倍行距,
    华文仿宋 小四号1.5倍行距,

    华文仿宋 小四号1.5倍行距,
    \subsection{节标题}
    
    华文仿宋 小四号1.5倍行距,
    \subsection{节标题}

    \begin{figure}[ht]
        \centering
        \includegraphics[scale=0.3]{pics/c1.jpg}
        \caption{ \zihao{5} \bf 图片示例}   
        \label{123} %%交叉索引 用 \ref{123} 引用
        \vspace{0.3cm}
    \end{figure}


    

 
    插入图片
    \begin{figure}[ht]
        \centering
        \includegraphics[scale=0.6]{pics/c1.jpg}
        \caption{ \zihao{5} \bf 图片及标题示例}    
        \vspace{0.4cm}
    \end{figure}
        
    插入表格
    \begin{table}[thp]
        \caption{ \zihao{5} \bf 表格示例}
        \zihao{5} \hwfs
        \vspace{0.2cm}
        \centering        
        \begin{tabular}{c|c|c}% 通过添加 | 来表示是否需要绘制竖线
            \hline  % 在表格最上方绘制横线
            xxx & XXX & xxx\\
            \hline  %在第一行和第二行之间绘制横线
            xxx & XXX & xxx\\
            \hline % 在表格最下方绘制横线        
        \end{tabular}
        \vspace{0.4cm}
   \end{table}

    公式示例
    \begin{equation}
        \nabla\cdot(\vec\sigma \nabla V)=0  in \Omega
   \end{equation}

\clearpage
}
\end{spacing}
\newpage