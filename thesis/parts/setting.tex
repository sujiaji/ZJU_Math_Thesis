\usepackage{amsmath}
\usepackage{amssymb}
\usepackage{amsthm}
\usepackage{cite}
\usepackage{extarrows}
\usepackage{indentfirst}
\usepackage{geometry}
\usepackage{multirow}
\usepackage{setspace}
\usepackage{fancyhdr} 
\usepackage{fontspec}
\usepackage{titlesec} 
\usepackage{titletoc}
\usepackage{pdfpages}
\usepackage{bm}
\usepackage{listings} 
\renewcommand{\thepart}{第\chinese{part}部分}
\renewcommand{\contentsname}{\centerline{ \hwfs \zihao{3} 目\quad 录}}
\pagestyle{fancy}
%%%%%%% 字体相关 %%%%%
\newcommand{\tnewroman}{\fontspec{Times New Roman}} %Times New Roman  tnewroman
\setCJKfamilyfont{hwxk}{STXingkai}   %华文行楷  hwxk  
\newcommand{\hwxk}{\CJKfamily{hwxk}} 
\setCJKfamilyfont{hwfs}{STFangsong}   %华文仿宋 hwfs 
\newcommand{\hwfs}{\CJKfamily{hwfs}} 
\newfontfamily\fangsong{FangSong}       %仿宋
\newcommand{\erhao}{\fontsize{21pt}{\baselineskip}\selectfont}

%%%% 页边距  %%%%%%%%%

\geometry{top=72pt,right=90pt,bottom=72pt,left=90pt} 

%%%%%%% 标题格式 %%%%%%%
    %%%%%   目录部分   %%%%%%
        % \part 部分相关
\titlecontents{part}[0pt]
    {\addvspace{1pt}\filright \bf \hwfs \zihao{4}}
    {\contentspush{\thecontentslabel  }}{}{}
        % \section 相关
\titlecontents{section}[0pt]{\zihao{-4} \hwfs }
    {\contentspush{\thecontentslabel ~}}{}
    {\titlerule*[6pt]{.}\contentspage}
        %\subsection 相关
\titlecontents{subsection}[0pt]{\zihao{-4} \hwfs }
    {\contentspush{\thecontentslabel ~}}{}
    {\titlerule*[6pt]{.}\contentspage}
        %\subsubsection 相关
\titlecontents{subsubsection}[0pt]{\zihao{-4} \hwfs }
    {\contentspush{\thecontentslabel ~}}{}
    {\titlerule*[6pt]{.}\contentspage}

    %%%%  文章内部  %%%%%
\titleformat{\part}%设置part的样式
    {\vspace{9cm}\centering \fontsize{48}{2} \heiti} %标签 48 磅 居中
    {\thepart }
    {0pt}%sep label和title之间的水平距离
    { \\ \centering \zihao{-0} \bf  \hwfs } % 标题仿宋GB, 加粗,小初居中

\titleformat{\section}%设置section的样式
    {\raggedright \zihao{3} \bfseries \hwfs}%右对齐,3号字,加粗
    {\thesection~}%标号后面有空格
    {0pt}%sep label和title之间的水平距离
    {}%标题前没有内容

\titleformat{\subsection}%设置subsection的样式
    {\raggedright \zihao{-3} \bfseries \hwfs }%右对齐,小3号字,加粗
    {\thesubsection~}%标号后面有个空格
    {0pt}%sep label和title之间的水平距离
    {}%标题前没有内容

\titleformat{\subsubsection}%设置subsubsection的样式
    {\raggedright \zihao{4} \bf \hwfs }%右对齐,4号字,加粗
    {\thesubsubsection~}%标号后面有个空格
    {0pt}%sep label和title之间的水平距离
    {}%标题前没有内容

%%%% 图、表、公式编号
\renewcommand{\thefigure}{\thesection.\arabic{figure}}
\renewcommand{\thetable}{\thesection.\arabic{table}}
\renewcommand\theequation{~ \arabic{section} - \arabic{equation} ~}
\renewcommand\arraystretch{1.5}
\renewcommand{\refname}{\vspace{-1cm}}

%%%%% 将当前页页眉重置为右侧页眉
\newcommand{\resetoddpageheadings}{  
    \ifodd\value{page}
        {
        \fancyhead{}            % 清空fancy设定
        \fancyhead[RO]{\zihao{-5} \biaoti}     %奇数页 右侧 页眉信息
        \fancyhead[LE]{\zihao{-5} 浙江大学本科生毕业论文(设计)}    %偶数页 左侧 页眉信息
        }
        \else
        {
        \fancyhead{}            % 清空fancy设定
        \fancyhead[RE]{\zihao{-5} \biaoti}     %奇数页 右侧 页眉信息
        \fancyhead[LO]{\zihao{-5} 浙江大学本科生毕业论文(设计)}    %偶数页 左侧 页眉信息
        }
    \fi
    }